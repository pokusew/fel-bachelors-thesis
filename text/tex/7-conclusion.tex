\chap[conclusion] Conclusion

In this thesis we dealt with the migration of the CTU's F1/10 autonomous driving stack from ROS~1 to ROS~2. First, we covered the necessary theory by describing both versions of ROS, "ros2_tracing" framework, and the components of the CTU's F1/10 stack. Then we migrated all the necessary parts so that we could demonstrate the Follow the Gap application running on ROS~2 on the real car and in the Stage simulator.

Then we spent a lot of time researching the ways to effectively evaluate runtime behavior of complex ROS~2 systems. We decided to use "ros2_tracing" and extend it with the message flow analysis based on the recent paper.
We did not have enough time to document all our achievements in this thesis, but they are all publicly available online on GitHub\urlnote{https://github.com/pokusew/fel-bachelors-thesis}.

Another result of our work is a publicly-available collection of setup guides, scripts, and documentation that covers various aspects of working with ROS. These guides have already helped several people.

We hope that the results of this thesis build a foundation that opens the way for the adoption of ROS~2 in CTU's F1/10 project.
