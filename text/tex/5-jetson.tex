\chap[jetson] NVIDIA Jetson TX2 Setup

In order to run ROS 2 on NVIDIA Jetson TX2, we needed to prepare an appropriate OS image.
In this part of the project, we worked with the NVIDIA Jetson TX2 Module on the NVIDIA Developer Kit carrier board.


\sec Initial Setup

NVIDIA provides {\em JetPack SDK} which is a collection of software for Jetson modules.
The current version is 4.5.1. It contains {\em L4T (Linux For Tegra)} 32.5.1 which consists of
flashing utilities, bootloader, Linux 4.9 Kernel, NVIDIA drivers and sample filesystem based on Ubuntu 18.04.

We used the NVIDIA SDK Manager to flash the most-up-to-date version of JetPack to our Jetson TX2 module.


\sec Building ROS 2 from Sources

Then we followed with the customization of the Jetson TX2's Ubuntu 18.04 system.
We had to built ROS 2 from sources as there are no binaries for Ubuntu 18.04 (only Ubuntu 20.04 is officially supported
by ROS 2 Foxy Fitzroy).\nl
The whole process is described at \url{https://github.com/pokusew/ros-setup/blob/master/nvidia-jetson-tx2/SETUP.md}.


\sec Bootable SD Card

Normally, the OS of the Jetson module is installed on its internal eMMC memory.
Sometimes, the size of the built-in eMMC memory (32 GB) is not enough.
At other times, one may want to be able to easily switch between different operating systems and configurations.
This is especially useful in environments, where one Jetson module is shared by many students and researches.

To support this use-case, Jetson TX2 can boot not only from its internal eMMC memory, but also from other media,
including an SD card.

In fact, NVIDIA Jetson TX2 has a relatively complicated 3-level boot process which is in detail described
at~\cite[l4t_tx2_boot_flow]. It is worth mentioning that all these three bootloaders and their configs are always
stored on hidden partitions on the internal eMMC memory and cannot be moved anywhere else.

Fortunately, we are only interested in the last boot phase that is handled by U-Boot.

“The U-Boot includes a default booting scan sequence. It scans bootable devices in the following order:
\begitems
* External SD card
* Internal eMMC
* USB device or NVMe device
* NFS network via DHCP/PXE
\enditems
U‑Boot looks for an "<rootfs>/boot/extlinux/extlinux.conf" configuration file of the bootable partition on each
bootable device”~\cite[l4t_uboot].

The Jetson TX2 (resp. the NVIDIA Developer Kit carrier board) supports SD cards up to SDR104 mode (UHS-1).

We purchased a SanDisk MicroSDXC 64GB Extreme Pro (with an SD adapter) which is the best possible SD card (UHS-1)
still supported by the Jetson TX2. We inserted the SD card into the Developer Kit SD card slot (while the Jetson TX2
was powered off as it does not support hot-plug). After powering the Jetson up, we formatted the card as "ext4" with
a single partition. To this partition we copied the previously configured root filesystem from the internal eMMC memory.
Then we updated the "/boot/extlinux/extlinux.conf" on the SD Card to incorporate new name of the root filesystem
device that is passed from U-Boot to the Linux Kernel\fnote{The resulting "extlinux.conf" can be found at
\url{https://github.com/pokusew/ros-setup/blob/master/nvidia-jetson-tx2/boot-config/mmcblk2/extlinux.conf}}.

Then we rebooted the Jetson. {\sbf This time it successfully booted from the SD card.}

This configuration works well for the use-case introduced above. {\sbf The whole system runs from the SD card}
("extlinux.conf" and the Linux Kernel Image are both loaded from the SD card). This way each Jetson TX2 user
can even customize the Linux Kernel Image for its application. When the SD card is not present during the boot,
the Jetson boots normally from its internal eMMC memory.

One might be interested also in performance impact on the system when running from the SD card compared to the
internal eMMC memory. We did not perform any performance impact measurements, but we did perform
a~{\em very simple}\fnote{During the test, a 10 MB of data was read 100 times.
\nl The screenshot of the result for {\sbf the eMMC} is here:
\nl \url{https://github.com/pokusew/ros-setup/blob/master/nvidia-jetson-tx2/boot-config/Screenshot\%20from\%202021-04-25\%2018-16-28.png}
\nl The screenshot of the result for {\sbf the SD card} is here:
\nl \url{https://github.com/pokusew/ros-setup/blob/master/nvidia-jetson-tx2/boot-config/Screenshot\%20from\%202021-04-25\%2019-45-06.png}
}
read speed test to get at least a rough estimate. We got the
following results:
\begitems
* 246.6 MB/s (the internal eMMC memory)
* 86.5 MB/s (the SanDisk MicroSDXC 64GB Extreme Pro)
\enditems
Please again note, that this test is mentioned just to get a rough estimate. A proper test (there might be big
differences in sequential/random accesses, OS caches impact, etc.) should be done.


\sec Running the Follow the Gap Demo

After we set up the ROS 2 environment on NVIDIA Jetson TX2, we wanted to demonstrate its working.
Though normally one would not use GUI on NVIDIA Jetson TX2, in this part of project we did decide to try it.
So we prepared also build of the Stage simulator for TX2's Ubuntu 18.
Then we successfully ran the Follow the Gap application
as can be seen in the Figure~\ref[img_follow_the_gap_ros2_stage_tx2]

\midinsert
\clabel[img_follow_the_gap_ros2_stage_tx2]{Follow the Gap in the Stage simulator on~NVIDIA Jetson~TX2}
\picw=14cm \cinspic ../images/follow_the_gap_ros2_stage_tx2.png
\caption/f Running the Follow the Gap application in the Stage simulator in ROS~2 on~NVIDIA Jetson~TX2
\endinsert
