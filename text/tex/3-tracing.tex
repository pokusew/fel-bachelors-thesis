\chap[f1tenth] CTU's F1/10 Platform

The F1/10 platform is a~scaled-down (1:10) model of an autonomous car that originates from F1/10 Autonomous
Racing Competition. Thanks to its affordability and similarity to a real car, the platform can be easily used for
development, testing and verification of autonomous driving systems and related algorithms. At CTU, multiple F1/10
models have been created and used ~\cite[vajnar_f1tenth_2017, dusil_detection_2019, klapalek_avoidance_2019]. {\em
Currently}, all of them are based on NVIDIA Jetson computing modules and powered by ROS~1 Kinetic Kame.

Throughout this project, we worked with the third model ("tx2-auto-3") which is based on NVIDIA Jetson TX2.
The model is shown in the Figure~\ref[img_tx2_auto_3]. The further descriptions of hardware stack are specific to
this model.

\midinsert
\clabel[img_tx2_auto_3]{The CTU's Third F1/10 Model Car}
\picw=8cm \cinspic ../images/tx2-auto-3.png
\caption/f The CTU's Third F1/10 Model Car based on NVIDIA Jetson TX2.
\endinsert


\sec Hardware Stack

The most notable components (with respect to this project) are:
\begitems
* NVIDIA Jetson TX2 Module – The central computing unit that runs the autonomous driving stack.
* Orbitty Carrier – A carrier board for the NVIDIA Jetson TX2 Module.
* VESC – Electric Speed Controller that control the engine.
* Teensy MCU – An auxiliary microcontroller for controlling the steering and handling the communication
with an RC Transmitter (Manual Control).
* Hokuyo UST-10LX – LIDAR
\enditems


\sec Software Stack

The software of stack is based on Ubuntu 16 (NVIDIA JetPack 3.x) and ROS~1 Kinetic Kame. The goal of this project
is to migrate the stack to ROS~2 Foxy Fitzroy. The migration comprises not only of porting the actual code to ROS~2
(which is done in Chapter~\ref[follow_the_gap]), but also of setting up the NVIDIA Jetson TX2 so that it can run ROS~2
(which is done in Chapter~\ref[jetson]).

The CTU F1/10 platform consists of multiple components that can be used in various combinations
to support different applications. A detailed overview of the CTU F1/10 platform architecture can be found
in the Section 4.3.2 of~\cite[klapalek_avoidance_2019].

Because migrating all the code of CTU F1/10 platform at once would not be smart, we need to select some part of
it that we will actually try to migrate. Such part should meet at least the following criteria:
\begitems
* It is possible to easily demonstrate its working in a simulator and on the real model.
* It contains minimal number of dependencies.
* It represents a typical autonomous driving application. That means reading data from sensor(s) (LIDAR), analyzing the
data (perception), planning a trajectory (decision and control) and controlling the vehicle.
\enditems

All these criteria all met by the {\em Follow the Gap} application which is a part of the CTU F1/10 platform.
It implements the Follow the Gap algorithm that was introduced in ~\cite[paper_follow_the_gap].
