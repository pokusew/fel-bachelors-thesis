\chap[conclusion] Conclusion

In this project we dealt with the migration of F1/10 autonomous driving stack from ROS~1 to ROS~2.

We described both versions of ROS and compared their differences. After introducing the CTU's F1/10 project,
we selected the Follow the Gap application for ROS~2 migration. We successfully migrated it to ROS~2
and demonstrated its working in the Stage simulator on Ubuntu and macOS.

Then we covered the setup of NVIDIA Jetson TX2 for running ROS~2 applications.
We also showed the Follow the Gap application running in Stage simulator on NVIDIA Jetson TX2.
We presented the possibility of booting from SD card which is useful in environments where the Jetson module
is shared among many students and researchers.

However, we did not run the Follow the Gap application on the real model car as the other steps consumed all the time
allotted for this project. Fortunately, only a~few additional (pretty straightforward) steps would be needed
to make it work (namely Orbitty Carrier setup and testing of ROS~2 version of LIDAR and VESC packages).

During this project, a~collection of setup guides and documentation was created. It covers various aspects
of working with ROS, especially the usage of ROS with IDEs (CLion and Visual Studio Code) which may be very useful
for other students.

The results of this project build a~foundation that opens the way for the adoption of ROS~2 in the CTU's F1/10 project.

% The results of this project form the first step in the effort to migrate the CTU's F1/10 project to ROS~2.

% The results of this project build a~foundation that opens the way for ROS~2 to CTU's F1/10 project.

