\chap[f1tenth] CTU's F1/10 Platform

The F1/10 platform is a scaled-down (1:10) model of an autonomous car that originates from F1/10 Autonomous Racing Competition (TODO: add ref or footnote). Thanks to its affordability and similarity to a real car, the platform can be easily used for the development, testing, and verification of autonomous driving systems and related algorithms.

At CTU, multiple F1/10 models have been created and used ~\cite[vajnar_f1tenth_2017, dusil_detection_2019, klapalek_avoidance_2019]. In this thesis, we focus on the latest iteration of the design represented by the models codenamed "tx2-auto-3"~\ref[img_tx2_auto_3] and "tx2-auto-usa"\footnote{This was designed by a fellow student, Tomáš Nagy. The main difference compared to the tx2-auto-3 is its chassis design (components' placement and mounting). The usa suffix in the model name refers to the fact that this model was assembled in the USA during our (my and Tomas Nagy's) internship at the University of Pennsylvania.}. This chapter aims to provide an up-to-date functional description of their hardware and software components. Such a description is a prerequisite for a successful migration to ROS 2 ~\ref[migration].

\midinsert
\clabel[img_tx2_auto_3]{The CTU's Third F1/10 Model Car}
\picw=8cm \cinspic ../images/tx2-auto-3.png
\caption/f The CTU's Third F1/10 Model Car based on NVIDIA Jetson TX2.
\endinsert


\sec Hardware Stack

The platform is based on an off-the-shelf RC car model from Traxxas (i.e., Traxxas Slash, but different ones can be used as well) with added components that enable autonomous operation (sensors, computers). The following table/list provides a quick overview of those components, which are then described in detail in the next sections. The diagram (ref diagram) shows the functional relationships among them.

\begitems
* NVIDIA Jetson TX2  – The main computing unit that runs the autonomous driving stack.
* VESC (Enertion FOCBOX VESC-X)– Electric Speed Controller (ESC) that controls the BLDC motor.
* Teensy 3.2 – A microcontroller (MCU) for controlling the steering and handling the communication with an RC Transmitter (Manual Control). It also implements an independent emergency stop (eStop).
* Hokuyo UST-10LX – LiDAR
* SparkFun 9DoF Razor IMU M0 – IMU
\enditems

\midinsert
\clabel[ctu_f1tenth_hw_diagram]{Functional Diagram}
\picw=8cm \cinspic ../images/ctu-f1tenth-v5-e.pdf
\caption/f Overview of main components with depicted methods of communication. The yellow blocks are part of the low-level system, which always works independently of the main high-level system (NVIDIA Jetson TX2).
\endinsert


\sec Software Stack

The software of stack is based on Ubuntu 16 (NVIDIA JetPack 3.x) and ROS~1 Kinetic Kame. The goal of this project
is to migrate the stack to ROS~2 Foxy Fitzroy. The migration comprises not only of porting the actual code to ROS~2
(which is done in Chapter~\ref[follow_the_gap]), but also of setting up the NVIDIA Jetson TX2 so that it can run ROS~2
(which is done in Chapter~\ref[jetson]).

The CTU F1/10 platform consists of multiple components that can be used in various combinations
to support different applications. A detailed overview of the CTU F1/10 platform architecture can be found
in the Section 4.3.2 of~\cite[klapalek_avoidance_2019].

Because migrating all the code of CTU F1/10 platform at once would not be smart, we need to select some part of
it that we will actually try to migrate. Such part should meet at least the following criteria:
\begitems
* It is possible to easily demonstrate its working in a simulator and on the real model.
* It contains minimal number of dependencies.
* It represents a typical autonomous driving application. That means reading data from sensor(s) (LIDAR), analyzing the
data (perception), planning a trajectory (decision and control) and controlling the vehicle.
\enditems

All these criteria all met by the {\em Follow the Gap} application which is a part of the CTU F1/10 platform.
It implements the Follow the Gap algorithm that was introduced in ~\cite[paper_follow_the_gap].
