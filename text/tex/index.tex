% CTUstyle3 template
\input ctustyle3

\worktype [B/EN]

\faculty {F3}
\department {Department of Cybernetics}

\title {Using ROS~2 for High-Speed Maneuvering in Autonomous Driving}
\titleCZ {Použití ROS~2 pro manévrování ve~vysoké rychlosti v~\nobreak{autonomním} řízení}
\author {Martin Endler}
\date {August 2022}

\supervisor {Ing. Michal Sojka, Ph.D.}
\studyinfo  {Open Informatics – Artificial Intelligence and Computer Science}
\workinfo   {\url{https://github.com/pokusew/fel-bachelors-thesis}}

% TODO: re-enable the assigment PDF for the final versions
% The two PDF Assigment pages after the Title Page
% \specification {
% 	\vbox to0pt{\vskip-32mm\centerline{\inspic ../../formalities/fel-bachelors-thesis-official-assignment-p1.pdf }\vss}
% 	\vfil\break
% 	\vbox to0pt{\vskip-32mm\centerline{\inspic ../../formalities/fel-bachelors-thesis-official-assignment-p2.pdf }\vss}
% }

% TODO: abstract, keywords

\abstractEN {

	The new version of the Robot Operating System – ROS~2 – brings significant improvements
	over its predecessor ROS~1. We briefly describe both versions and compare their differences.
	Then we cover the process of migrating a ROS 1 application to ROS~2 on the example of the Follow the Gap app
	that is part of CTU's F1/10 project.

	In the next part, we focus on setting up an NVIDIA Jetson TX2 module so that it can run ROS 2 applications and
	provide good developer experience in environments where Jetson module is shared among many students and
	researchers.

	The first result of this project is a~ROS~2 port of the Follow the Gap application. Its working is
	demonstrated in the Stage simulator on Ubuntu and macOS. The second result is a setup guide for NVIDIA Jetson
	TX2 that covers creating a fully-setup OS image with ROS 2 on a bootable SD card. The last result is a
	collection of setup guides and documentation that cover various aspects of working with ROS.

}
\keywordsEN {
	ROS, ROS 1, ROS 2, ROS 2 migration, F1/10, Follow the Gap, autonomous model car, NVIDIA Jetson TX2
}

\abstractCZ {

	Tady bude abstrakt v českém jazyce.

}
\keywordsCZ {
	Tady budou klíčová slová v čekém jazyce.
}

\thanks {
	First of all, I would like to thank my~supervisor Ing.~Michal~Sojka,~Ph.D.
	\nl for his guidance, support, and immense patience.

	Second, I would like to thank Ing.~Jaroslav Klapálek for providing valuable
	information and advice regarding the F1/10 project.

	Next, I would like to thank Bc.~Tomáš Nagy for his help with all sorts of mechanical stuff,
	explaining AV algorithms, building a new CTU's F1/10 model,
	and for all his support during our joint internship at the \nobreak{University} of Pennsylvania.

	Last but not least, I would like to thank Prof.~Rahul~Mangharam
	for \nobreak{providing} a stimulating environment during my internship
	at his lab at the \nobreak{University} of Pennsylvania.

	Finally, I would like to thank my \nobreak{family} and close friends for always supporting me throughout my studies.
}
\declaration {
	I declare that the presented work was developed independently and that I have listed all sources of information
	used within it in accordance with the methodical instructions for observing the ethical principles in the
	preparation of university theses.

	Prague, August 15, 2022
	\signature % makes dots
}

% link glossary data
\input glossary

%%%%% <--   % The place for your own macros is here.

%\draft     % Uncomment this if the version of your document is working only.
%\linespacing=1.7  % uncomment this if you need more spaces between lines
% Warning: this works only when \draft is activated!
%\savetoner        % Turns off the lightBlue backround of tables and
% verbatims, only for \draft version.
%\blackwhite       % Use this if you need really Black+White thesis.
%\onesideprinting  % Use this if you really don't use duplex printing.

% make title page, acknowledgment, contents etc.
\makefront

% contents
\input 1-introduction
\input 2-ros
\input 3-f1tenth
\input 4-follow-the-gap
\input 5-jetson
\input 6-conclusion

% appendices
\input appendices

% document end
\bye
