\chap[follow_the_gap] Migration of the Follow the Gap

We started by analyzing the operation of the Follow the Gap in its original ROS~1 environment.
Then we also examined its code. We identified the packages that needed to be migrated to ROS~2.
There are three main parts (nodes) that power the application:
\begitems
* "perception/recognition/obstacle_substitution" – It converts data from LIDAR to obstacles.
* "decision_and_control/follow_the_gap_v0" – It consists of two nodes. First plans the trajectory using the data
about obstacles.
While the second converts the plan to the control commands and sends them to the Drive-API.
* "vehicle_platform/drive_api" – It accepts control commands and controls the car accordingly (speed and steering).
\enditems

Apart from the nodes, there are messages definitions, launch files and various configuration that needs to be
migrated too.

After we got familiar with the structure of the application in ROS~1 environment, we actually proceeded
with the migration itself. It was not an easy task as there were lots of tricky details and issues that needed
to be solved. It required a~lof of searching in the official documentation as well as examining the ROS~2 source
code and examples as not all concepts were equally well documented.

Nevertheless, we {\sbf succeeded} in the task and we {\sbf successfully migrated} all the code required to run the
application in a~simulator. The code can be found
in "f1tenth_rewrite"\urlnote{https://github.com/pokusew/f1tenth-rewrite} repository.

The most notable issue that has arisen is the fact that ROS~2 has no global Parameter Server.
Instead, in ROS~2, parameters are managed per node. Each node implements a~parameter service that allows getting and
setting the parameters during runtime. Initial values of parameters can be supplied at node startup time via command
line arguments (parameters file). One of the most useful advantages this design brings, is that all parameters
can be easily changed during runtime and nodes can implement custom behavior for handling the changes.
In ROS~1, one must use solutions like "dynamic_reconfigure"\urlnote{https://wiki.ros.org/dynamic_reconfigure}
to get similar features.

We benefited from the new parameters' possibilities in the "follow_the_gap_v0_ride" node where we got rid of the
"dynamic_reconfigure".

On the other hand, in some situations it might be useful to have a~“global parameter storage”.
In fact, the CTU's F1/10 project uses this concept very much. However, even in ROS~2, we can implement a~node that
will act as a “global parameter storage” and will store the global car model configuration (such as sensors' types)
so that the launch configurations can be shared among different car models.


\sec Demonstration in the Stage simulator

To verify the working of the migrated application, we used the Stage simulator. The Stage simulator was introduced
in~\cite[klapalek_avoidance_2019]. We used "stage_ros2"\urlnote{https://github.com/ymd-stella/stage_ros2} package as
bridge between ROS~2 system and the Stage simulator. However, the installation of Stage simulator
was not so straightforward as it was not present in package managers on Ubuntu or macOS.
So we had to build it from sources. The exact steps were different for each platform. As a~part of this project,
a~colcon workspace "ros2-build"\urlnote{https://github.com/pokusew/ros2-build} was created that (among other things)
allows building and using the Stage simulator.

\midinsert
\clabel[img_follow_the_gap_ros2_stage_ubuntu]{Follow the Gap in the Stage simulator on~Ubuntu}
\picw=14cm \cinspic ../images/follow_the_gap_ros2_stage_ubuntu.png
\caption/f Running the Follow the Gap application in the Stage simulator in ROS~2 on~Ubuntu~20.04 (as a~VM on~macOS)
\endinsert

\midinsert
\clabel[img_follow_the_gap_ros2_stage_macos]{Follow the Gap in the Stage simulator on~macOS}
\picw=14cm \cinspic ../images/follow_the_gap_ros2_stage_macOS.png
\caption/f Running the Follow the Gap application in the Stage simulator in ROS~2 \hbox{on~macOS~10.14.6}
\endinsert
